\documentclass{article}
\usepackage{graphicx} % Required for inserting images
\usepackage[spanish]{babel}
\usepackage{multirow}

\title{ Trabajo Práctico \\ 
    \vspace{15}
    \textit{Introducci\'on a routing} \\
}
\author{Nicolás Fagetti}
\date{\today}

\begin{document}
\maketitle

\textbf{
Configuracion de las PCs: \\
------------------------------------------------------------------------ \\
}
\begin{verbatim}
    C:\> ipconfig 198.162.0.2 255.255.255.0
    C:\>ipconfig

FastEthernet0 Connection:(default port)

   Connection-specific DNS Suffix..: 
   Link-local IPv6 Address.........: FE80::290:2BFF:FE42:18E9
   IPv6 Address....................: ::
   IPv4 Address....................: 198.162.0.2
   Subnet Mask.....................: 255.255.255.0
   Default Gateway.................: ::
                                     0.0.0.0

Bluetooth Connection:

   Connection-specific DNS Suffix..: 
   Link-local IPv6 Address.........: ::
   IPv6 Address....................: ::
   IPv4 Address....................: 0.0.0.0
   Subnet Mask.....................: 0.0.0.0
   Default Gateway.................: ::
                                     0.0.0.0

    C:\> ipconfig 198.162.1.2 255.255.255.0
    C:\>ipconfig

FastEthernet0 Connection:(default port)

   Connection-specific DNS Suffix..: 
   Link-local IPv6 Address.........: FE80::201:42FF:FE7C:2AC
   IPv6 Address....................: ::
   IPv4 Address....................: 198.162.1.2
   Subnet Mask.....................: 255.255.255.0
   Default Gateway.................: ::
                                     0.0.0.0

Bluetooth Connection:

   Connection-specific DNS Suffix..: 
   Link-local IPv6 Address.........: ::
   IPv6 Address....................: ::
   IPv4 Address....................: 0.0.0.0
   Subnet Mask.....................: 0.0.0.0
   Default Gateway.................: ::
                                     0.0.0.0
                                     
\end{verbatim}
\textbf{
Configuracion de puertos GigabitEthernet del Router: \\
------------------------------------------------------------------------ \\
}
\begin{verbatim}
    Router0>enable
    Router0#configure terminal
    Enter configuration commands, one per line.  End with CNTL/Z.
    Router0(config)#interface gigabitEthernet 0/0
    Router0(config-if)#ip address 192.168.0.1 255.255.255.0
    Router0(config-if)#no shutdown

    Router0(config-if)#
    %LINK-5-CHANGED: Interface GigabitEthernet0/0, changed state to up
    %LINEPROTO-5-UPDOWN: Line protocol on Interface GigabitEthernet0/0, changed state to up

    Router0(config)#interface gigabitEthernet 0/1
    Router0(config-if)#ip address 192.168.1.1 255.255.255.0
    Router0(config-if)#no shutdown

    Router0(config-if)#
    %LINK-5-CHANGED: Interface GigabitEthernet0/1, changed state to up
    %LINEPROTO-5-UPDOWN: Line protocol on Interface GigabitEthernet0/1, changed state to up

    Router0(config)#do show ip route
Codes: L - local, C - connected, S - static, R - RIP, M - mobile, B - BGP
       D - EIGRP, EX - EIGRP external, O - OSPF, IA - OSPF inter area
       N1 - OSPF NSSA external type 1, N2 - OSPF NSSA external type 2
       E1 - OSPF external type 1, E2 - OSPF external type 2, E - EGP
       i - IS-IS, L1 - IS-IS level-1, L2 - IS-IS level-2, ia - IS-IS inter area
       * - candidate default, U - per-user static route, o - ODR
       P - periodic downloaded static route

Gateway of last resort is not set

     192.168.0.0/24 is variably subnetted, 2 subnets, 2 masks
C       192.168.0.0/24 is directly connected, GigabitEthernet0/0
L       192.168.0.1/32 is directly connected, GigabitEthernet0/0
     192.168.1.0/24 is variably subnetted, 2 subnets, 2 masks
C       192.168.1.0/24 is directly connected, GigabitEthernet0/1
L       192.168.1.1/32 is directly connected, GigabitEthernet0/1

\end{verbatim}

\newpage
\section{Ruteo b\'asico de host}
\subsection{Configuraci\'on de host}

Si enviamos un ping de la PC0 a la PC1 (en otra red), el mismo no sale de la PC0. Esto sucede porque la direcci\'on de destino indicada no se encuentra dentro de la misma subred, y tampoco es una direcci\'on de broadcast. Adem\'as, el Default Gateway no est\'a seteado, por lo que no sabe a d\'onde enviar el paquete con destino desconocido. \\

Al cambiar la m\'ascara en PC0, si enviamos el mismo ping veremos que env\'ia un ARP que viaja a trav\'es del switch hasta el Router y retorna a la PC0 con \'exito. Luego, env\'ia un ICMP que tambi\'en viaja a trav\'es del switch pero al llegar al Router, es descartado. Esto sucede porque el Router, al recibir el mensaje, busca la direcci\'on IP destino en la tabla CEF y no la encuentra, y luego la busca en la tabla de ruteo, donde la encuentra e intenta ubicar el pr\'oximo salto. Como este nodo no se encuentra en la tabla ARP, se env\'ia un mensaje ARP y el paquete se descarta. Este mensaje ARP que se env\'ia al destino tras descartar el paquete de origen, es retornado al Router con \'exito. \\

Al configurar la Default Gateway para la PC0, si enviamos el ping veremos que se vuelve a enviar el ARP hacia el Router igual que en el caso anterior, pero el ICMP tambi\'en viaja hacia el Router y \'este env\'ia un paquete unicast hacia la PC1, el cual intentar\'a responder, pero al no reconocer a la direcci\'on de destino de la respuesta (PC0) como parte de la subred o direcci\'on de broadcast, y no poseer configurado su Default Gateway, descartar\'a el mensaje ICMP. \\

\subsection{Default gateway}
Configuramos el Default Gateway para ambas PCs a trav\'es del siguiente comando ampliado: \\\\
(PC 0)
\begin{verbatim}
    C:\>ipconfig 192.168.0.2 255.255.255.0 192.168.0.1
\end{verbatim}
(PC 1)
\begin{verbatim}
    C:\>ipconfig 192.168.1.2 255.255.255.0 192.168.1.1
\end{verbatim}

Si enviamos el ping nuevamente desde PC0 a PC1, veremos que primeramente enviar\'a un paquete ARP y luego enviar\'a el ICMP correctamente (en rigor, el primer intento de env\'io falla, ya que el Router primero descarta el paquete hasta haber realizado el env\'io de un ARP a la PC1). \\

\section{Los routers}
\subsection{Forwarding vs routing}
\subsection{Static routing}
Tras configurar las direcciones ip de los hosts y los routers, adem\'as del clock rate del extremo DCE de la conexi\'on serial entre ambos routers, verificamos que al enviar los pings entre PC0 y PC1, y entre PC2 y PC3, respectivamente, el env\'io es exitoso. \\

Sin embargo, al intentar enviar desde PC0 a PC3, por ejemplo, el env\'io falla, sin poder continuar m\'as all\'a del Router0. Agregamos entonces nuevas rutas est\'aticamente, con los siguientes comandos:
\begin{verbatim}
    Router0(config)#ip route 192.168.2.0 255.255.255.0 192.168.5.1
    Router0(config)#ip route 192.168.3.0 255.255.255.0 192.168.5.1

    Router1(config)#ip route 192.168.0.0 255.255.255.0 192.168.5.2
    Router1(config)#ip route 192.168.1.0 255.255.255.0 192.168.5.2
    
\end{verbatim}

Verificamos que la configuraci\'on sea exitosa, con el comando \texttt{show ip route}: \\
\begin{verbatim}
Router1(config)#do show ip route
Codes: L - local, C - connected, S - static, R - RIP, M - mobile, B - BGP
       D - EIGRP, EX - EIGRP external, O - OSPF, IA - OSPF inter area
       N1 - OSPF NSSA external type 1, N2 - OSPF NSSA external type 2
       E1 - OSPF external type 1, E2 - OSPF external type 2, E - EGP
       i - IS-IS, L1 - IS-IS level-1, L2 - IS-IS level-2, ia - IS-IS inter area
       * - candidate default, U - per-user static route, o - ODR
       P - periodic downloaded static route

Gateway of last resort is not set

S    192.168.0.0/24 [1/0] via 192.168.5.2
S    192.168.1.0/24 [1/0] via 192.168.5.2
     192.168.2.0/24 is variably subnetted, 2 subnets, 2 masks
C       192.168.2.0/24 is directly connected, GigabitEthernet0/0
L       192.168.2.1/32 is directly connected, GigabitEthernet0/0
     192.168.3.0/24 is variably subnetted, 2 subnets, 2 masks
C       192.168.3.0/24 is directly connected, GigabitEthernet0/1
L       192.168.3.1/32 is directly connected, GigabitEthernet0/1
     192.168.5.0/24 is variably subnetted, 2 subnets, 2 masks
C       192.168.5.0/30 is directly connected, Serial0/1/0
L       192.168.5.1/32 is directly connected, Serial0/1/0

\end{verbatim}
Algo similar veremos en el Router0: \\
\begin{verbatim}
Router0(config)# do show ip route
Codes: L - local, C - connected, S - static, R - RIP, M - mobile, B - BGP
       D - EIGRP, EX - EIGRP external, O - OSPF, IA - OSPF inter area
       N1 - OSPF NSSA external type 1, N2 - OSPF NSSA external type 2
       E1 - OSPF external type 1, E2 - OSPF external type 2, E - EGP
       i - IS-IS, L1 - IS-IS level-1, L2 - IS-IS level-2, ia - IS-IS inter area
       * - candidate default, U - per-user static route, o - ODR
       P - periodic downloaded static route

Gateway of last resort is not set

     192.168.0.0/24 is variably subnetted, 2 subnets, 2 masks
C       192.168.0.0/24 is directly connected, GigabitEthernet0/0
L       192.168.0.1/32 is directly connected, GigabitEthernet0/0
     192.168.1.0/24 is variably subnetted, 2 subnets, 2 masks
C       192.168.1.0/24 is directly connected, GigabitEthernet0/1
L       192.168.1.1/32 is directly connected, GigabitEthernet0/1
S    192.168.2.0/24 [1/0] via 192.168.5.1
S    192.168.3.0/24 [1/0] via 192.168.5.1
     192.168.5.0/24 is variably subnetted, 2 subnets, 2 masks
C       192.168.5.0/30 is directly connected, Serial0/1/0
L       192.168.5.2/32 is directly connected, Serial0/1/0

\end{verbatim}

Como vemos, las nuevas rutas han sido correctamente agregadas, cosa que podremos constatar al hacer nuevamente el env\'io entre los diferentes hosts. Los mensajes ICMP viajan sin problemas a trav\'es de ambos routers y llegan a su destino correctamente. \\

Para sumarizar las redes contiguas en una nueva que las contenga, realizamos el siguiente proceso: \\

1- Convertimos las direcciones contiguas a binario:
\begin{verbatim}
    192.168.0.0 --> 11000000 10101000 00000000 00000000
    192.168.1.0 --> 11000000 10101000 00000001 00000000
    192.168.2.0 --> 11000000 10101000 00000010 00000000
    192.168.3.0 --> 11000000 10101000 00000011 00000000
    
\end{verbatim}

2- Verificamos los bits coincidentes en ambas direcciones de izquierda a derecha. El prefijo obtenido es el siguiente: \\

\texttt{11000000 10101000 000000} \\

El prefijo obtenido posee 22 bits, por lo que el sufijo de la nueva direcci\'on sumarizada ser\'a ``/22''. \\

3- Completamos el resto de bits con 0s, hasta completar los 32 bits, y lo convertimos a decimal: \\

\texttt{11000000 10101000 000000\textit{00 00000000} --> 192.168.0.0} \\

4- Le agregamos el sufijo obtenido en el paso 2: \\

\texttt{192.168.0.0/22} \\

Si queremos, podemos convertir el sufijo en m\'ascara decimal: \\

22 bits = \texttt{11111111.11111111.11111100.00000000 = 255.255.252.0} \\\\

Una vez obtenida la nueva direcci\'on sumarizada, podemos reconfigurar las rutas: \\
\begin{verbatim}
Router0(config)#ip route 192.168.0.0 255.255.252.0 192.168.5.1
Router0(config)#do show ip route
Codes: L - local, C - connected, S - static, R - RIP, M - mobile, B - BGP
       D - EIGRP, EX - EIGRP external, O - OSPF, IA - OSPF inter area
       N1 - OSPF NSSA external type 1, N2 - OSPF NSSA external type 2
       E1 - OSPF external type 1, E2 - OSPF external type 2, E - EGP
       i - IS-IS, L1 - IS-IS level-1, L2 - IS-IS level-2, ia - IS-IS inter area
       * - candidate default, U - per-user static route, o - ODR
       P - periodic downloaded static route

Gateway of last resort is not set

S    192.168.0.0/22 [1/0] via 192.168.5.1
     192.168.0.0/24 is variably subnetted, 2 subnets, 2 masks
C       192.168.0.0/24 is directly connected, GigabitEthernet0/0
L       192.168.0.1/32 is directly connected, GigabitEthernet0/0
     192.168.1.0/24 is variably subnetted, 2 subnets, 2 masks
C       192.168.1.0/24 is directly connected, GigabitEthernet0/1
L       192.168.1.1/32 is directly connected, GigabitEthernet0/1
S    192.168.2.0/24 [1/0] via 192.168.5.1
S    192.168.3.0/24 [1/0] via 192.168.5.1
     192.168.5.0/24 is variably subnetted, 2 subnets, 2 masks
C       192.168.5.0/30 is directly connected, Serial0/1/0
L       192.168.5.2/32 is directly connected, Serial0/1/0

\end{verbatim}

Como se puede apreciar, la nueva entrada sumarizada aparece correctamente, pero las entradas anteriores contin\'uan registradas. Para eliminarlas, ingresamos el siguiente comando: \\
\begin{verbatim}
Router0(config)#no ip route 192.168.2.0 255.255.255.0 192.168.5.1
Router0(config)#do show ip route
Codes: L - local, C - connected, S - static, R - RIP, M - mobile, B - BGP
       D - EIGRP, EX - EIGRP external, O - OSPF, IA - OSPF inter area
       N1 - OSPF NSSA external type 1, N2 - OSPF NSSA external type 2
       E1 - OSPF external type 1, E2 - OSPF external type 2, E - EGP
       i - IS-IS, L1 - IS-IS level-1, L2 - IS-IS level-2, ia - IS-IS inter area
       * - candidate default, U - per-user static route, o - ODR
       P - periodic downloaded static route

Gateway of last resort is not set

S    192.168.0.0/22 [1/0] via 192.168.5.1
     192.168.0.0/24 is variably subnetted, 2 subnets, 2 masks
C       192.168.0.0/24 is directly connected, GigabitEthernet0/0
L       192.168.0.1/32 is directly connected, GigabitEthernet0/0
     192.168.1.0/24 is variably subnetted, 2 subnets, 2 masks
C       192.168.1.0/24 is directly connected, GigabitEthernet0/1
L       192.168.1.1/32 is directly connected, GigabitEthernet0/1
S    192.168.3.0/24 [1/0] via 192.168.5.1
     192.168.5.0/24 is variably subnetted, 2 subnets, 2 masks
C       192.168.5.0/30 is directly connected, Serial0/1/0
L       192.168.5.2/32 is directly connected, Serial0/1/0

\end{verbatim}

Como vemos, la entrada \texttt{192.168.2.0} ya no aparece. Haremos lo mismo con la otra entrada, y repetiremos el proceso en el otro router: \\
\begin{verbatim}
Router0(config)#no ip route 192.168.3.0 255.255.255.0 192.168.5.1

Router1(config)#ip route 192.168.0.0 255.255.252.0 192.168.5.2
Router1(config)#no ip route 192.168.0.0 255.255.255.0 192.168.5.2
Router1(config)#no ip route 192.168.1.0 255.255.255.0 192.168.5.2
Router1(config)#do show ip route
Codes: L - local, C - connected, S - static, R - RIP, M - mobile, B - BGP
       D - EIGRP, EX - EIGRP external, O - OSPF, IA - OSPF inter area
       N1 - OSPF NSSA external type 1, N2 - OSPF NSSA external type 2
       E1 - OSPF external type 1, E2 - OSPF external type 2, E - EGP
       i - IS-IS, L1 - IS-IS level-1, L2 - IS-IS level-2, ia - IS-IS inter area
       * - candidate default, U - per-user static route, o - ODR
       P - periodic downloaded static route

Gateway of last resort is not set

S    192.168.0.0/22 [1/0] via 192.168.5.2
     192.168.2.0/24 is variably subnetted, 2 subnets, 2 masks
C       192.168.2.0/24 is directly connected, GigabitEthernet0/0
L       192.168.2.1/32 is directly connected, GigabitEthernet0/0
     192.168.3.0/24 is variably subnetted, 2 subnets, 2 masks
C       192.168.3.0/24 is directly connected, GigabitEthernet0/1
L       192.168.3.1/32 is directly connected, GigabitEthernet0/1
     192.168.5.0/24 is variably subnetted, 2 subnets, 2 masks
C       192.168.5.0/30 is directly connected, Serial0/1/0
L       192.168.5.1/32 is directly connected, Serial0/1/0

\end{verbatim}

De esta forma, queda reconfigurada la ruta entre redes. \\

Para configurar el default gateway en los routers, ingresamos el prefix 0.0.0.0 0.0.0.0 a trav\'es del comando \texttt{ip route}: \\
\begin{verbatim}
Router0(config)#ip route 0.0.0.0 0.0.0.0 192.168.5.1
Router0(config)#do show ip route
Codes: L - local, C - connected, S - static, R - RIP, M - mobile, B - BGP
       D - EIGRP, EX - EIGRP external, O - OSPF, IA - OSPF inter area
       N1 - OSPF NSSA external type 1, N2 - OSPF NSSA external type 2
       E1 - OSPF external type 1, E2 - OSPF external type 2, E - EGP
       i - IS-IS, L1 - IS-IS level-1, L2 - IS-IS level-2, ia - IS-IS inter area
       * - candidate default, U - per-user static route, o - ODR
       P - periodic downloaded static route

Gateway of last resort is 192.168.5.1 to network 0.0.0.0

S    192.168.0.0/22 [1/0] via 192.168.5.1
     192.168.0.0/24 is variably subnetted, 2 subnets, 2 masks
C       192.168.0.0/24 is directly connected, GigabitEthernet0/0
L       192.168.0.1/32 is directly connected, GigabitEthernet0/0
     192.168.1.0/24 is variably subnetted, 2 subnets, 2 masks
C       192.168.1.0/24 is directly connected, GigabitEthernet0/1
L       192.168.1.1/32 is directly connected, GigabitEthernet0/1
     192.168.5.0/24 is variably subnetted, 2 subnets, 2 masks
C       192.168.5.0/30 is directly connected, Serial0/1/0
L       192.168.5.2/32 is directly connected, Serial0/1/0
S*   0.0.0.0/0 [1/0] via 192.168.5.1

\end{verbatim}

De la informaci\'on volcada, podemos verificar que el ``last resort gateway'' (gateway de \'ultimo recurso) se encuentra ahora seteado, y aparece una nueva entrada marcada con S*, que indica dicha ruta. \\

\subsection{Redes stubs}
En el escenario de 4 hosts, se puede decir que tenemos 4 redes stub, ya que todas se comunican con las otras redes mp contiguas a trav\'es de un \'unico gateway. \\

En las redes stub, cada host tendr\'a un IP diferente, todos compartir\'an la misma m\'ascara, y todos tendr\'an una entrada en su tabla de ruteo para poder acceder a hosts fuera de su red, es decir, un default gateway. \\

Si ampl\'io el escenario, por ejemplo, mejorando la conectividad agregando un nuevo Router2 conectado al Router1, tendremos una nueva interfaz conectada al Router1, por lo que deberemos definir por cu\'al de las 2 interfaces debo enviar el tr\'afico hacia las otras redes. \\

\newpage
\section{Referencias}
\begin{itemize}
    \item https://www.cyberghostvpn.com/es/glossary/stub-network
    \item https://community.cisco.com/t5/discusiones-routing-y-switching/sumarizaci\'on/td-p/3394672
    \item https://github.com/moro1982/tp-intro2routing
\end{document}
